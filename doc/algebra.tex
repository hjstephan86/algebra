\documentclass[oneside]{scrbook}
\usepackage[a4paper, total={6in, 10in}]{geometry}
\usepackage[ngerman]{babel}

\usepackage{xcolor}
\usepackage{graphicx}
\usepackage{multirow} % Wichtig: Fügen Sie dieses Paket in Ihrer Präambel hinzu!
\usepackage{array} % Oft nützlich in Verbindung mit tabularx

\usepackage{amsmath}
\usepackage{amsthm}
\usepackage{amssymb}
\usepackage{algorithm}
\usepackage[noend]{algpseudocode}

\usepackage{tikz}

\usepackage[colorlinks=true, linkcolor=black, citecolor=black, urlcolor=black]{hyperref}

\newtheorem{satz}{Satz}[section]
\newtheorem{lemma}{Lemma}[section]
\newtheorem{definition}{Definiton}[section]
\numberwithin{equation}{section}

\setcounter{secnumdepth}{3} % Ensures subsubsection and above are numbered

\makeatletter
\renewcommand{\ALG@name}{Algorithmus}
\makeatother
\algrenewcommand\algorithmicrequire{\textbf{Eingabe:}}
\algrenewcommand\algorithmicensure{\textbf{Ausgabe:}}

\title{Grundlagen der Algebra}
\author{Stephan Epp\\\texttt{hjstephan86@gmail.com}}
\date{\today}

\begin{document}
	\maketitle
	\vspace{5em}
	\tableofcontents
\chapter{Einleitung}
Was ist Algebra? Von seiner Wortbedeutung her bedeutet Algebra die \textit{Lehre von den Gleichungen} oder die \textit{Theorie der Verknüpfungen mathematischer Strukturen}. Um Aussagen darüber treffen zu können, was Gleichungen oder Verknüpfungen sind, muss definiert werden, wie die Strukturen aussehen, die miteinander verglichen oder verknüpft werden. Um den Begriff Struktur besser zu verstehen, lohnt sich ein Blick in die Informatik. Mögliche Strukturen in der Informatik sind zum Beispiel \textit{Listen} oder \textit{Stacks}. Dabei enthält ein Liste eine Menge von Elementen, wobei die Elemente jeweils einen Wert annehmen. Die Beschreibung der Liste als Datenstruktur eignet sich aber auch schon für die Beschreibung der mathematischen Struktur des \textit{Vektors}. Mögliche Verknüpfungen von Vektoren sind zum Beispiel die Addition oder Multiplikation zweier Vektoren, die in dieser Verknüpfung wieder einen Vektor als Ergebnis haben. Es ist nicht unüblich in der Informatik auch eine Liste von Listen zu nutzen, um zur Laufzeit effizient auf die Datenstruktur zugreifen zu können. Die Beschreibung der Liste von Listen eignet sich aber auch für die Beschreibung der mathematischen Struktur der \textit{Matrix}. Mögliche Verknüpfungen von Matrizen sind zum Beispiel die Addition oder Multiplikation zweier Matrizen, die in dieser Verknüpfung wieder eine Matrix als Ergebnis haben.
\chapter{Definitionen}
Zur Betrachtung der Algebra folgen Definitionen, die für den weiteren Verlauf dieser Arbeit nützlich sind.
\section{Strukturen}
\begin{definition}
	Ein Vektor $\boldsymbol{v} = (v_1, \ldots, v_n)$ aus dem Raum $\mathbb{R}^n$ hat die Größe $n$ und ist ein Tupel von $n$ Elementen, wobei jedes Element $v_i \in \mathbb{R}$.
\end{definition}
Zu beachten ist, dass der Vektor $\boldsymbol{v}$ fett geschrieben ist. Zum Beispiel ist $\boldsymbol{0}$ der Vektor, bei dem alle Elemente den Wert null haben.
\begin{definition}
	Eine Matrix $A$ aus dem Raum $\mathbb{R}^{n \times m}$ hat die Größe $n \times m$ und besteht aus $n$ Zeilen und $m$ Spalten, wobei jedes Element $a_{ij} \in \mathbb{R}$.
\end{definition}
\begin{definition}
   	Zwei Vektoren $\boldsymbol{u}$ und $\boldsymbol{v}$ projezieren den Vektor $\boldsymbol{w}$ genau dann, wenn es Konstanten $k_1$ und $k_2$ gibt, so dass $$k_1 \boldsymbol{u} + k_2\boldsymbol{v} = \boldsymbol{w} \neq \boldsymbol{0},$$ dabei haben $\boldsymbol{u}$, $\boldsymbol{v}$ und $\boldsymbol{w}$ dieselbe Größe und $k_i \in \mathbb{R}$, $k_i \neq 0$.
\end{definition}
Das heißt, die Vektoren $\boldsymbol{u}$ und $\boldsymbol{v}$ bilden eine \textit{Projektion} $\boldsymbol{w}$ in Abhängigkeit der Konstanten $k_1$, $k_2$, wobei die Konstanten nicht den Wert null haben.
\begin{definition}
	Für die Projektion $\boldsymbol{w}$ durch die Vektoren $\boldsymbol{u}$ und $\boldsymbol{v}$ liegen $\boldsymbol{u}$ und $\boldsymbol{v}$ in der Umgebung von $\boldsymbol{w}$ im Raum $\mathbb{R}^n$, wobei $\boldsymbol{u}$, $\boldsymbol{v}$ und $\boldsymbol{w}$ jeweils Größe $n$ haben.
\end{definition}
Nicht alle Vektoren $\boldsymbol{u}$ und $\boldsymbol{v}$ liegen in der Umgebung von $\boldsymbol{w}$. Es gibt Vektoren im Raum $\mathbb{R}^n$, durch die $\boldsymbol{w}$ niemals projeziert werden kann.
\begin{definition}
	Eine Projektion $\boldsymbol{w}$ durch die Vektoren $\boldsymbol{u}$ und $\boldsymbol{v}$ und Konstanten $k_1$, $k_2$ ist minimal, wenn für alle Konstanten $k_i$ gilt, wird eine Konstante $k_i = 0$, dann ist $$k_1 \boldsymbol{u} + k_2\boldsymbol{v} = \boldsymbol{0}.$$
\end{definition}
Das bedeutet, wenn eine minimale Projektion gefunden wurde, entferne die eine Konstante $k$, d.h., $k = 0$, und erhalte mit $k_1 \boldsymbol{u} + k_2\boldsymbol{v} = \boldsymbol{0}$ die \textit{größte Einheit}, mit der, egal wie ihre Vektoren miteinander kombiniert werden, nichts mehr projeziert werden kann außer $\boldsymbol{0}$. Damit wurde eine größte und nicht mehr projezierbare Einheit gefunden, der \textit{Kern}.

Der Kern in seiner Umgebung des Raumes ist nicht teilbar, er kann nicht weiter reduziert werden. Der \textit{triviale Kern} besteht nur aus den Vektoren, bei denen jeweils nur ein Element den Wert eins hat, sonst haben alle anderen Elemente den Wert null.
\begin{definition}
	Die Einheitsmatrix $E$ ist gegeben durch den trivialen Kern in entsprechender Ordnung, $$ E = (\boldsymbol{e_1} \: \boldsymbol{e_2} \: \boldsymbol{e_3} \: \boldsymbol{e_4}) = 
	\begin{pmatrix}
		1 & 0 & 0 & 0 \\
		0 & 1 & 0 & 0 \\
		0 & 0 & 1 & 0 \\
		0 & 0 & 0 & 1
	\end{pmatrix},$$
wobei $E$ die Größe vier hat und die Vektoren $\boldsymbol{e_1}, \ldots, \boldsymbol{e_4}$  jeweils die Größe vier haben.
\end{definition}
Es ist beachten, dass bei den Vektoren $\boldsymbol{e_1}, \ldots, \boldsymbol{e_4}$ jeweils nur ein Element den Wert eins hat, sonst haben alle anderen Elemente den Wert null. Außerdem gilt:
$$ k_1\boldsymbol{e_1} + k_2\boldsymbol{e_2} + k_3\boldsymbol{e_3} + k_4\boldsymbol{e_4} = \boldsymbol{0},$$ egal, welchen Wert die Konstanten $k_i \in \mathbb{R} $ annehmen. In dieser Ordnung $\boldsymbol{e_1}, \ldots, \boldsymbol{e_4}$ bilden sie die Einheitsmatrix und auch einen Kern im Raum $\mathbb{R}^{4 \times 4}$. 

\section{Verknüpfungen}
Es ist bekannt, dass bei der Vektoraddition oder -multiplikation zwei Vektoren im Allgemeinen nicht miteinander addiert oder multipliziert werden können. Zwei Vektoren können zum Beispiel nur dann miteinander multipliziert werden, wenn sie dieselbe Größe haben. Erst dann ergibt das Produkt der beiden Vektoren wieder einen Vektor derselben Größe.

Der \textit{Bezug} zwischen den Vektoren einer Umgebung ist durch die Verknüpfung, d.h., die Operation wie der Addition oder Multiplikation, im Ergebnis der Operation gegeben. Ist das Ergebnis der Operation $\boldsymbol{0}$, dann gibt es keinen Bezug. Ist das Ergebnis der Operation $\boldsymbol{w}$ und $\boldsymbol{w} \neq \boldsymbol{0}$, dann gibt es einen Bezug. Dieser Bezug oder, allgemeiner, diese Beziehungen werden veranschaulicht in den bekannten Operationen für Vektoren, das sind die Vektoraddition und die Vektormultiplikation. Insbesondere wird der Bezug aber auch beschrieben durch die Projektion $\boldsymbol{w}$ durch $\boldsymbol{u}$ und $\boldsymbol{v}$.

\begin{definition}
	\label{def:vektor-summe}
	Die Summe zweier Vektoren $\boldsymbol{u} = (u_1, \ldots, u_n)$ und $\boldsymbol{v} = (v_1, \ldots, v_n)$ ist gegeben durch $$\boldsymbol{u} + \boldsymbol{v} = (u_1 + w_1, \ldots, u_n + w_n) = \boldsymbol{w},$$ wobei $\boldsymbol{u}$, $\boldsymbol{v}$ und $\boldsymbol{w}$ jeweils Größe $n$ haben.
\end{definition}
\begin{definition}
	\label{def:vektor-produkt}
	Die Produkt zweier Vektoren $\boldsymbol{u} = (u_1, \ldots, u_n)$ und $\boldsymbol{v} = (v_1, \ldots, v_n)$ ist gegeben durch $$\boldsymbol{u} \cdot \boldsymbol{v} = (u_1 \cdot w_1, \ldots, u_n \cdot w_n) = \boldsymbol{w},$$ wobei $\boldsymbol{u}$, $\boldsymbol{v}$ und $\boldsymbol{w}$ jeweils Größe $n$ haben.
\end{definition}
Es fällt auf, dass die Verknüpfung der Addition für die Vektoren $\boldsymbol{u}$ und $\boldsymbol{v}$ für die Verknüpfung der Multiplikation durch die Addition ersetzt wird. Bei der Definition der Addition und der Multiplikation sind die Elemente $v_i$ der Vektoren reelle Zahlen, d.h., $v_i \in \mathbb{R}$. Diese Definition der Addition und der Multiplikation für die Vektoren $\boldsymbol{u}$ und $\boldsymbol{v}$ gilt aber auch, wenn die Elemente $v_i$ der Vektoren binäre Zahlen sind, d.h., $v_i \in \{0, 1\}$. Dabei steht die Summe der Vektoren für die logische ODER-Verknüpfung, kurz $\lor$, und das Produkt der Vektoren für die logische UND-Verknüpfung, kurz $\land$.

Es ist bekannt, dass bei der Matrixaddition oder -multiplikation zwei Matrizen im Allgemeinen nicht miteinander addiert oder multipliziert werden können. Zwei Matrizen können zum Beispiel nur dann miteinander multipliziert werden, wenn die Anzahl der Spalten der ersten Matrix gerade die Anzahl der Zeilen der zweiten Matrix ist. Erst dann ergibt das Produkt der beiden Matrizen wieder eine Matrix mit entsprechender Größe. Die Anzahl der Zeilen der Produktmatrix ist die Anzahl der Zeilen der ersten Matrix und die Anzahl der Spalten der Produktmatrix ist die Anzahl der Spalten der zweiten Matrix.

\begin{definition}
	Die Summe zweier Matrizen $A$ und $B$ ist gegeben durch 
	$$A + B = (\boldsymbol{a_1} \: \ldots \: \boldsymbol{a_m}) + (\boldsymbol{b_1} \: \ldots \: \boldsymbol{b_m}) = (\boldsymbol{a_1} + \boldsymbol{b_1} \: \ldots \: \boldsymbol{a_m} + \boldsymbol{b_m}) = C,$$ wobei $A$ eine $n \times m$ Matrix, $B$ eine $n \times m$ Matrix und $C$ eine $n \times m$ Matrix ist.
\end{definition}
Die Matrix $C$ als Summe von $A$ und $B$ wird gebildet durch die paarweise Addition $(\boldsymbol{a_1} + \boldsymbol{b_1} \: \ldots \: \boldsymbol{a_m} + \boldsymbol{b_m})$ der Vektoren $(\boldsymbol{a_1} \: \ldots \: \boldsymbol{a_m})$ und $(\boldsymbol{b_1} \: \ldots \: \boldsymbol{b_m})$.
\begin{definition}
	Das Produkt zweier Matrizen $A$ und $B$ ist gegeben durch 
	$$A \cdot B = C = (c_{ij}), \text{ mit } c_{ij} = \sum_{k=1}^{r}a_{ik}b_{kj},$$ wobei $A$ eine $n \times r$ Matrix, $B$ eine $r \times m$ Matrix und $C$ eine $n \times m$ Matrix ist.
\end{definition}
Die Produktmatrix $C$ mit $C = (c_{ij})$ wird gebildet durch die Summe $\sum_{k=1}^{r}a_{ik}b_{kj}$ über alle $1 \leq i \leq n$ und alle $1 \leq j \leq m$ für $c_{ij}$. Dabei werden für das Element $c_{ij}$ genau $r$ Summanden $a_{ik}b_{kj}$ über alle $1 \leq k \leq r$ addiert.

\section{Körper}
Warum sind Vektoren in ihrer Struktur so wohl geformt in dem Sinn, dass die Verknüpfung der Addition einfach durch die Multiplikation oder die Verknüpfung der Multiplikation einfach durch die Addition ersetzt werden kann? Die Vektoren bilden einen Körper mit einfachen Eigenschaften. Matrizen haben diese Eigenschaft in der Verknüpfung nicht. Die Addition kann nicht einfach durch die Multiplikation ersetzt werden, da die Struktur der Matrix aufwendiger ist in der Anwendung der Multiplikation. Der Aufwand für eine einfache Vorgehensweise zur Multiplikation von zwei Matrizen hat eine Laufzeit von $O(n^3)$, wenn zwei $n \times n$ Matrizen multipliziert werden. Vektoren dagegen können paarweise multipliziert werden. Der Aufwand für eine einfache Vorgehensweise zur Multiplikation von zwei Vektoren hat im Vergleich dazu eine Laufzeit von $O(n)$. Es wurde bereits darauf eingegangen, dass die Verknüpfungen (Addition und Multiplikation) bei Vektoren sowohl für Elemente aus den reellen als auch den binären Zahlen gelten.

Wie verhalten sich die Vektoren hinsichtlich des neutralen Elements und des inversen Elements? Wie verhalten sich Vektoren im Bereich der komplexen Zahlen? Welche Vorteile hat es, dass Vektoren die Struktur eines Körpers haben? Eignen sich die Vektoren aus algebraischer und analytischer Sicht besonders? Geht es mehr um die Instanzen dieses Körpers, welche die konkreten Vektoren dann sind und damit eine angenehme Behandlung hinsichtlich maximal vieler Verknüpfungen auf ihnen ermöglichen? Für welchen Wertebereich sind Instanzen des Körpers immer noch Instanzen des Körpers, nur für die reellen Zahlen und die binären Zahlen? Ist es das Ziel beim Finden von charakteristische Eigenschaften eines Körpers besonders viele charakteristische Eigenschaften zu finden? 
\begin{satz}
	Diese charakteristischen Eigenschaften bilden einen wohl geformten Körper: (1) die Verknüpfungen: Addition und Multiplikation und (2) der Wertebereich.
\end{satz}
\begin{proof}
	Klar ist, je mehr charakteristische Eigenschaften ein Körper hat, desto weniger wahrscheinlich ist es, dass es Instanzen dieses Körpers auf möglichst vielen Verknüpfungen und Wertebereichen gibt. Bei der Untersuchung dieser Frage geht es um die Verknüpfungen und die Wertebereiche. Denn in ihrer Kombination wird das Ergebnis bestimmt und sichtbar, ob eine Instanz des Körpers noch Instanz des Körpers ist. Es muss also um Verknüpfungen und Wertebereiche gehen. Werden die Verknüpfungen nicht betrachtet, lässt sich kein Ergebnis berechnen, da die Instanzen miteinander noch nicht mal zu einem Ergebnis kombiniert werden. Wird der Wertebereich nicht betrachtet, lässt sich auch kein Ergebnis in einem konkreten Wertebereich wie den reellen oder binären Zahlen berechnen.

	Es reicht für einen wohl geformten Körper, die Verknüpfungen und die Wertebereiche zu betrachten. Werden weitere Eigenschaften wie das neutrale oder das inverse Element als charakteristische Eigenschaften betrachtet, geht dies mit mehr Aufwand in der Anwendung der Verknüpfungen für solche Instanzen einher. Dies wird zum Beispiel daran deutlich, dass der Körper, wie ursprünglich definiert, die 1 als neutrales Element hat aber der Ring, wie ursprünglich definiert, nicht.
	
	Die Differenz zwischen 1 und 0 ist die größte, es gibt keine größere Differenz. Dass für das Betrachten der größten Differenz nur benachbarte ganze Zahlen relevant sind, geht daraus hervor, dass nur die kleinsten, nicht weiter teilbaren Zahlen dazu in Betracht kommen und erst darin ihr eigentlicher Wert durch Vergleichen klar wird. Die Differenz zwischen 2 und 1 ist nicht so groß. Warum? Egal, mit welcher anderen Zahl außer 0 die 0 vervielfacht wird, das Ergebnis der Verfielfachung ist wieder 0. Bei der 1 ist das anders. Wird die 1 mit 2 vervielfacht, ist das Ergebnis der Verfielfachung 2. Das ist eigenartig und stellt die Frage, warum erscheint die Vervielfachung als angemessenere Verknüpfung in ihrer Wirkung als die einfache Addition? Die Verfielfachung ist angenehmer, weil mit ihr ein größerer Ergebnisraum erreicht wird als mit der Addition. Die Addition kann auf der anderen Seite durch die Multiplikation nicht ersetzt werden. Dies begründet, dass die Addition und die Multiplikation beides notwendige Verknüpfungen sind für einen wohl geformten Körper.
\end{proof}

\begin{satz}
	Die 1 ist die wichtigste Zahl aller ganzen Zahlen. 
\end{satz}
\begin{proof}
	Die Differenz zwischen 0 und 1 ist größer ist als die zwischen 2 und 1, aber die Differenz zwischen 3 und 2 ist nicht größer als die zwischen 2 und 1. Dies gilt für alle folgenden Zahlen von 2, 3, 4 usw. Damit ist klar, dass 0 und 1 die wichtigsten Zahlen sind, denn ihre Differenz ist unter allen ganzen Zahlen die größte. Da 0 aber keinen Wert hat, außer den, den sie der 1 in der Differenz gibt, ist von ihnen die 1 die wichtigere Zahl.
\end{proof}
\begin{lemma}
	Die Vektoren bilden einen wohl geformten Körper für reelle oder binäre Zahlen.
\end{lemma}
\begin{proof}
	Der Beweis geht daraus hervor, dass Vektoren diese charakteristischen Eigenschaften haben: (1) Die Verknüpfungen wie  Addition und Multiplikation wie in Definition \ref{def:vektor-summe} und Definition \ref{def:vektor-produkt} definiert für die (2) Wertebereiche der reellen oder binären Zahlen.
\end{proof}
\begin{satz}
	Die Algebra ist Grundlage für die Analyse.
\end{satz}
\begin{proof}
Ohne Algebra macht die Analyse keinen Sinn, denn ohne Wertebereiche, ob für natürliche, reelle oder komplexe Zahlen, lässt sich keine analytische Aussage über das Ergebnis einer Verknüpfung machen.
\end{proof}

Wenn Matrizen hinsichtlich der Matrixmultiplikation eine interessante Anwendung für Graphen haben, wird das auch auf jeden Fall gelten für den wohl geformten Körper der Vektoren. Betrachtet man bei der Multiplikation zweier Matrizen, dass die Addition durch $\min(\ldots)$, das Minimum, und die Multiplikation durch die Addition ersetzt wird, kann das All-Pairs-Shortest-Paths (APSP) Problem gelöst werden. Der Algorithmus \ref{alg:apsp-mm} berechnet in der Matrix $C$ die kürzeste Distanz für alle Knoten $i, j$ des Graphen unter Verwendung dieser modifizierten Matrixmultiplikation.
\begin{algorithm}
	\caption{APSP($A, B$)}
	\label{alg:apsp-mm}
	\begin{algorithmic}[1]
		\Require $\langle A, B\rangle$ mit $n \times n$ Matrix $A$ und $B$
		\Ensure $\langle C\rangle$ mit $n \times n$ Matrix $C$, die die Längen der kürzesten Wege aller $(i, j)$ enthält
		\For{$i = 1 \text{ to } n$}
		\For{$j = 1 \text{ to } n$}
		\For{$k = 1 \text{ to } n$}
		\If{$A[i][k] \neq \infty \text{ und } B[k][j] \neq \infty$}
		\State $C[i][j] = \min(C[i][j], A[i][k] + B[k][j])$
		\EndIf
		\EndFor
		\EndFor
		\EndFor
		\State \Return $C$
	\end{algorithmic}
\end{algorithm} Werden die Kanten des Graphen mit der Adjazenzmatrix $A$ beschrieben, dann wird Algorithmus \ref{alg:apsp-mm} mit APSP($A, A$) aufgerufen. Zur Ermittlung von $C$ benötigt Algorithmus \ref{alg:apsp-mm} eine Laufzeit von $O(n^3)$. Es lohnt sich also für den wohl geformten Körper der Vektoren effiziente Algorithmen zu entwickeln.

\chapter{Anwendungen}
\label{chap:anwendungen}
Zur Anwendung der grundlegenden Definitionen der Strukturen und Verknüpfungen werden in diesem Kapitel konkrete Vorgehensweisen zur Verknüpfung der Strukturen beschrieben.

\section{Matrixmultiplikation}
Dieser Abschnitt beschreibt Vorgehensweisen zur Verknüpfung der Multiplikation auf der Struktur der Matrix. Dabei wird angenommen, dass zwei $n \times n$ Matrizen $A$ und $B$ multipliziert werden.

\subsection{Herkömmliche Multiplikation}
Die herkömmliche Matrixmultiplikation ist gegeben durch ihre Definition und benötigt in der Anwendung eine Laufzeit von $O(n^3)$, wenn wir der Einfachheit davon ausgehen, dass zwei $n \times n$ Matrizen miteinander multipliziert werden.
\begin{algorithm}
	\caption{Matrixmultiplikation($A, B$)}
	\label{alg:mm}
	\begin{algorithmic}[1]
		\Require $\langle A, B\rangle$ mit $n \times n$ Matrix $A$ und $B$
		\Ensure $\langle C\rangle$ mit $n \times n$ Matrix $C = A \cdot B$
		\For{$i = 1 \text{ to } n$}
		\For{$j = 1 \text{ to } n$}
		\For{$k = 1 \text{ to } n$}
		\State $C[i][j] = C[i][j] + (A[i][k] \cdot B[k][j])$
		\EndFor
		\EndFor
		\EndFor
		\State \Return $C$
	\end{algorithmic}
\end{algorithm}
Algorithmus \ref{alg:mm} zeigt die Vorgehensweise zur Berechnung der Produktmatrix $C = A \cdot B$.

\subsection{Verbesserte Multiplikation}
Es folgt der Algorithmus \ref{alg:strassen} von \textsc{Strassen}  als Pseudocode. Als Eingabe erhalten wir zwei $n \times n$ Matrizen. Der Einfachheit halber wird angenommen, dass $n$ eine Zweierpotenz ist.
\begin{algorithm}
	\caption{\textsc{Strassen}$(A, B)$}
	\label{alg:strassen}
	\begin{algorithmic}[1]
		\Require $\langle A, B \rangle$ mit $n \times n$ Matrizen $A$, $B$, $n = 2^k$, $k \in \mathbb{N}$
		\Ensure $\langle C \rangle$ mit Produktmatrix $C = AB$
		\If{$n = 1$} \textbf{return} $C = AB$
		\EndIf
		%\Comment{Matrizen in $n/2 \times n/2$ Blöcke unterteilen}
		\State $A = \begin{pmatrix} A_{11} & A_{12} \\ A_{21} & A_{22} \end{pmatrix}$
		\State $B = \begin{pmatrix} B_{11} & B_{12} \\ B_{21} & B_{22} \end{pmatrix}$
		%\Comment{Berechne die 7 Produkte rekursiv}
		\State $P_1 = \textsc{Strassen}(A_{11} + A_{22}, B_{11} + B_{22})$
		\State $P_2 = \textsc{Strassen}(A_{21} + A_{22}, B_{11})$
		\State $P_3 = \textsc{Strassen}(A_{11}, B_{12} - B_{22})$
		\State $P_4 = \textsc{Strassen}(A_{22}, B_{21} - B_{11})$
		\State $P_5 = \textsc{Strassen}(A_{11} + A_{12}, B_{22})$
		\State $P_6 = \textsc{Strassen}(A_{21} - A_{11}, B_{11} + B_{12})$
		\State $P_7 = \textsc{Strassen}(A_{12} - A_{22}, B_{21} + B_{22})$
		% \Comment{Berechne die Blöcke von $C$}
		\State $C_{11} = P_1 + P_4 - P_5 + P_7$
		\State $C_{12} = P_3 + P_5$
		\State $C_{21} = P_2 + P_4$
		\State $C_{22} = P_1 - P_2 + P_3 + P_6$
		\State \textbf{return} $C = \begin{pmatrix} C_{11} & C_{12} \\ C_{21} & C_{22} \end{pmatrix}$
	\end{algorithmic}
\end{algorithm}
Zu Beginn prüfen wir, ob die Größe der Matrizen bereits den Wert $1$ hat. Haben die Matrizen den Wert $1$, geben wir das Produkt $A B$ zurück. In Zeile 2 und 3 werden die Matrizen $A$ und $B$ so definiert, dass in den Zeilen 4 bis 10 die 5 Matrixmultiplikationen jeweils durchgeführt werden. In den Zeilen 11 bis 14 werden 4 Matrizen $C_{ij}$ durch Addition und Subtraktion der Matrizen $A_{ij}$ berechnet und in Zeile 15 Matrix $C$ als Ergebnis zurückgegeben.

Als Idee zum Beweis der Korrektheit betrachten wir das Produkt $A \cdot B$ der zwei Matrizen $A$ und $B$ mit
\begin{align*}
	A = \begin{pmatrix} a & b \\ c & d \end{pmatrix} \quad \text{und} \quad B = \begin{pmatrix} e & f \\ g & h \end{pmatrix}.
\end{align*}
Zur Vereinfachung beinhalten die Matrizen nur skalare Werte. Das Ergebnis $C = A \cdot B$ ist
\begin{align*}
	C = \begin{pmatrix} ae + bg & af + bh \\ ce + dg & cf + dh \end{pmatrix}.
\end{align*}
Der Algorithmus von \textsc{Strassen} berechnet $P_r$ mit
\[
\begin{array}{ll}
	P_1 = \textsc{Strassen}(a + d, e + h) & = ae + ah + de + dh, \\
	P_2 = \textsc{Strassen}(c + d, e)     & = ce + de, \\
	P_3 = \textsc{Strassen}(a, f - h)     & = af - ah, \\
	P_4 = \textsc{Strassen}(d, g - e)     & = dg - de, \\
	P_5 = \textsc{Strassen}(a + b, h)     & = ah + bh, \\
	P_6 = \textsc{Strassen}(c - a, e + f) & = ce + cf - ae - af, \\
	P_7 = \textsc{Strassen}(b - d, g + h) & = bg + bh - dg - dh.
\end{array}
\]
Dann werden $C_{ij}$ berechnet mit
\[
\begin{array}{ll}
	C_{11} = P_1 + P_4 - P_5 + P_7 & = ae + bg, \\
	C_{12} = P_3 + P_5 & = af + bh, \\
	C_{21} = P_2 + P_4 & = ce + dg, \\
	C_{22} = P_1 - P_2 + P_3 + P_6 & = cf + dh,
\end{array}
\]
wobei z.B. $ C_{11} = (ae + ah + de + dh) + (dg - de) - (ah + bh) + (bg + bh - dg - dh) = ae + bg$. Für einen formalen Beweis der Korrektheit verzichten wir auf die vollständige Induktion über $n \in \mathbb{N}$ der $n \times n$ Matrizen.

\subsubsection{Laufzeitanalyse}
Für die Multiplikation von zwei $n \times n$ Matrizen lautet die Rekurrenzgleichung für den Algorithmus von \textsc{Strassen}
\begin{align*}
	T(n) = 7 \: T(\tfrac{n}{2}) + c\: n^2.
\end{align*}
Der Algorithmus halbiert in jedem rekursiven Aufruf die beiden $n \times n$ Matrizen zu vier $\tfrac{n}{2} \times \tfrac{n}{2}$ Matrizen. Substituieren wir $n$ durch $\tfrac{n}{2}$, erhalten wir für
\begin{align*}
	T(\tfrac{n}{2}) &= 7 \: T(\tfrac{n}{4}) + c(\tfrac{n}{2})^2 = 7 \: T(\tfrac{n}{4}) + \tfrac{c}{4}n^2.
\end{align*}
Nach dem \textit{ersten} rekursiven Aufruf erhalten wir mit $T(\frac{n}{2})$ eingesetzt in $T(n)$ dann
\begin{align*}
	T(n) &= 7 \: (7 \: T(\tfrac{n}{4}) + \tfrac{c}{4} \: n^2) + c n^2 \\
	&= 7^2\: T(\tfrac{n}{4}) + \tfrac{7}{4} \: cn^2 + c n^2.
\end{align*}
Mit dem \textit{zweiten} rekursiven Aufruf werden die vier $\tfrac{n}{2} \times \tfrac{n}{2}$ Matrizen wieder halbiert zu acht $\tfrac{n}{4} \times \tfrac{n}{4}$ Matrizen. Damit ist
\begin{align*}
	T(\tfrac{n}{4}) &= 7 \: T(\tfrac{n}{8}) + c(\tfrac{n}{4})^2 = 7 \: T(\tfrac{n}{8}) + \tfrac{c}{16}n^2.
\end{align*}
Wird $T(\tfrac{n}{4})$ eingesetzt in $T(n)$ ergibt sich
\begin{align*}
	T(n) &= 7^2\: (7 \: T(\tfrac{n}{8}) + \tfrac{c}{16}n^2) + \tfrac{7}{4}cn^2 + c n^2 \\
	&= 7^3\: T(\tfrac{n}{2^3}) + \tfrac{7^2}{4^2}cn^2 + \tfrac{7}{4}cn^2 + c n^2.
\end{align*}
Betrachten wir nun den $k$-ten rekursiven Aufruf finden wir für 
\begin{align*}
	T(n) = 7^k \: T \: \big(\tfrac{n}{2^k}\big) + cn^2 \: \sum_{i = 0}^{k-1} \bigg(\frac{7}{4}\bigg)^i.
\end{align*}
Zur Vereinfachung belassen wir es bei dem $k$-ten rekursiven Aufruf auch in $T(n)$ bei $k$ und nicht $k + 1$. Kleinere Matrizen als $1 \times 1$ Matrizen gibt es nicht, daher können die $n \times n$ Matrizen nur $k$ mal halbiert werden. Der größte Wert, den $k$ annehmen kann, ist $k = \log_{2}n$. Damit ist
\begin{align*}
	\vphantom{\rule{0pt}{2.5ex}} T(n) &= 7^{\log_{2}n} \: T\Big(\frac{n}{2^{\log_{2}n}}\Big) + cn^2 \: \sum_{i = 0}^{\log_{2}n-1} \left(\tfrac{7}{4}\right)^i \\
	\vphantom{\rule{0pt}{2.5ex}} &= n^{\log_{2}7} \: T(1) + cn^2 \:\frac{\left(\tfrac{7}{4}\right)^{\log_{2}n} - 1}{\tfrac{7}{4} - 1} \\
	\vphantom{\rule{0pt}{2.5ex}} &= O\left(n^{2.8074}\right) + cn^2 \left(\tfrac{7}{4}\right)^{\log_{2}n} \\
	\vphantom{\rule{0pt}{2.5ex}} &= O\left(n^{2.8074}\right) + cn^2 \cdot n^{\log_{2}\tfrac{7}{4}} \\
	\vphantom{\rule{0pt}{2.5ex}} &= O\left(n^{2.8074}\right) + cn^{2.8074} \\
	\vphantom{\rule{0pt}{2.5ex}} &= O\left(n^{2.8074}\right).
\end{align*}

\subsection{Optimale Multiplikation}
Um die Anzahl der benötigten Matrixmultiplikationen von 7 auf 5 zu reduzieren, wird die Diagonale $e$ und $h$ der Matrix $B$ betrachtet. Dadurch wird für $n$ eine Laufzeit im Exponenten von $2.3219$ statt $2.807$ erreicht. Betrachtet man $P_2 = (c + d) \cdot e = ce + de$ und $P_5 = (a + b) \cdot h = ah + bh$, dann fällt auf, dass $P_2$ durch $P_6$ und $P_1$ ermittelt werden kann,
\begin{align*}
	ce &= P_6 - cf + ae + af, \\
	de &= P_1 - ae - ah - dh,
\end{align*}
und $P_5$ durch $P_1$ und $P_7$
\begin{align*}
	ah &= P_1 - ae - de - dh, \\
	bh &= P_7 - bg + dg + dh.
\end{align*}
Dazu müssen $P_1$, $P_6$ und $P_7$ berechnet werden bevor $P_2$ und $P_5$ ohne Multiplikation bestimmt werden können:
\begin{align*}
	P_2 &= ce + de = (P_6 - cf + ae + af) + (P_1 - ae - ah - dh) = P_6 + P_1 - cf + af - ah + af, \\
	P_5 &= ah + bh = (P_1 - ae - de - dh) + (P_7 - bg + dg + dh) = P_1 + P_7 - ae - de - bg + dg.
\end{align*}
Da nur 5 Matrixmultiplikationen verwendet werden, ergibt sich für die Laufzeitanalyse 
\begin{align*}
	\vphantom{\rule{0pt}{2.5ex}} T(n) &= 5^{\log_{2}n} \: T\Big(\frac{n}{2^{\log_{2}n}}\Big) + cn^2 \sum_{i = 0}^{\log_{2}n-1} \left(\tfrac{5}{4}\right)^i
	\vphantom{\rule{0pt}{2.5ex}} = n^{\log_{2}5} \: T(1) + cn^2 \frac{\left(\tfrac{5}{4}\right)^{\log_{2}n} - 1}{\tfrac{5}{4} - 1}
	\vphantom{\rule{0pt}{2.5ex}} = O\left(n^{2.3219}\right).
\end{align*}
Dazu wird der Algorithmus von \textsc{Strassen} zu \textsc{Strassen-25} so geändert, dass $P_2$ und $P_5$ in Zeile 9 und 10 ermittelt werden. Damit $P_2$ und $P_5$ ermittelt werden können, müssen die zuvor ermittelten Produkte $P_r$, $r \in \{1, 6, 7\}$ verwendet werden. Das bedeutet für 
\begin{align*}
	P_2 &= P_6 + P_1 - A_{21}B_{12} + A_{11}B_{12} - A_{11}B_{22} + A_{11}B_{12}, \\
	P_5 &= P_1 + P_7 - A_{11}B_{11} - A_{22}B_{11} - A_{21}B_{21} + A_{22}B_{21}.
\end{align*}
Dabei ist $P_2$ \textit{abhängig} von $P_6$ und $P_7$ und $P_5$ ist \textit{abhängig} von $P_1$ und $P_7$. Damit $P_2$ und $P_5$ ohne Matrixmultiplikation ermittelt werden können, müssen die Produkte $A_{ij}B_{ij}$ im vorherigen Rekursionsschritt jeweils gespeichert werden.
\begin{algorithm}
	\caption{\textsc{Strassen-25}$(A, B)$}
	\label{alg:strassen25}
	\begin{algorithmic}[1]
		\Require $\langle A, B \rangle$ mit $n \times n$ Matrizen $A$, $B$, $n = 2^k$, $k \in \mathbb{N}$
		\Ensure $\langle C, A_{ij}, B_{ij} \rangle$ mit Produktmatrix $C = AB$ und $A_{ij}$, $B_{ij}$
		\If{$n = 1$} \textbf{return} $C = AB$
		\EndIf
		%\Comment{Matrizen in $n/2 \times n/2$ Blöcke unterteilen}
		\State $A = \begin{pmatrix} A_{11} & A_{12} \\ A_{21} & A_{22} \end{pmatrix}$
		\State $B = \begin{pmatrix} B_{11} & B_{12} \\ B_{21} & B_{22} \end{pmatrix}$
		%\Comment{Berechne die 7 Produkte rekursiv}
		\State $P_1 = \textsc{Strassen-25}(A_{11} + A_{22}, B_{11} + B_{22})$
		\State $P_3 = \textsc{Strassen-25}(A_{11}, B_{12} - B_{22})$
		\State $P_4 = \textsc{Strassen-25}(A_{22}, B_{21} - B_{11})$
		\State $P_6 = \textsc{Strassen-25}(A_{21} - A_{11}, B_{11} + B_{12})$
		\State $P_7 = \textsc{Strassen-25}(A_{12} - A_{22}, B_{21} + B_{22})$
		\State $P_2 = P_6 + P_1 - A_{21}B_{12} + A_{11}B_{12} - A_{11}B_{22} + A_{11}B_{12}$ 
		\State $P_5 = P_1 + P_7 - A_{11}B_{11} - A_{22}B_{11} - A_{21}B_{21} + A_{22}B_{21}$
		% \Comment{Berechne die Blöcke von $C$}
		\State $C_{11} = P_1 + P_4 - P_5 + P_7$
		\State $C_{12} = P_3 + P_5$
		\State $C_{21} = P_2 + P_4$
		\State $C_{22} = P_1 - P_2 + P_3 + P_6$
		\State \textbf{return} $C = \begin{pmatrix} C_{11} & C_{12} \\ C_{21} & C_{22} \end{pmatrix}$
	\end{algorithmic}
\end{algorithm}
% Bei der Implementierung von \textsc{Strassen-25}$(A, B)$ müssen die Produkte $A_{ij}B_{ij}$ in jedem Rekursionsschritt für die gesamte Rekursion in einer Datenstruktur gespeichert werden. Dabei muss auf die Datenstruktur mit einem eindeutigen Schlüssel $(k, i_{A}, j_{A}, i_{B}, j_{B})$ in $O(1)$ zugegriffen werden, wobei $n = 2^k$, $k \in \mathbb{N}$.
Damit in jedem Rekursionsschritt die Produkte $A_{ij}B_{ij}$ effizient gespeichert werden können, muss auf eine Datenstruktur mit einem eindeutigen Schlüssel $(k, i_{A}, j_{A}, i_{B}, j_{B})$ in $O(1)$ zugegriffen werden, $n = 2^k$, $k \in \mathbb{N}$.
\begin{satz}
	Es gibt keinen Teile-und-Herrsche Algorithmus zur Multiplikation von $n \times n$ Matrizen, der durch Nutzen von Abhängigkeiten weniger als 8 Matrixmultiplikationen benötigt und damit auch weniger als 5 Matrixmultiplikationen benötigt. 
\end{satz}
\begin{proof}
	Angenommen, es gibt einen Teile-und-Herrsche Algorithmus zur Multiplikation von $n \times n$ Matrizen, der mit weniger als 5 Matrixmultiplikationen auskommt, dann muss dieser Algorithmus auch Abhängigkeiten $P_r = (P_s, P_t, A_{ij}B_{ij})$ verwenden, mit denen $P_r$ ermittelt werden kann, $r < 5$. Da $dim(A_{ij}) = dim(B_{ij}) = \frac{n}{2}$, ist es nicht möglich $C = AB$ zu berechnen. Das gesamte Bild der \textit{projezierten} Produktmatrix $C$ kann nur projeziert werden, wenn mindestens mehr als die Hälfte beider Matrizen $A_{ij}$ und $B_{ij}$ zum Projezieren verfügbar ist.
\end{proof}
\begin{satz}
	Das optimale Teile-und-Herrsche Prinzip ist die effizienteste Methode, um Probleme zu lösen, wenn die Probleminstanz bei jedem rekursiven Aufruf halbiert wird. 
\end{satz}
\begin{proof}
	In diesen drei Kriterien ist der Beweis zu führen: (1) Rekursionstiefe, (2) Anzahl der Teilprobleme, (3) Größe eines Teilproblems. Dabei wird schnell klar, dass es alles in der Halbierung zusammengefasst optimal ist. Denn die Halbierung ist nichts anderes als eine Verdoppelung im Nenner.
	
	Zu (1): Wichtig in der Analyse der Laufzeit eines Algorithmus oder einer Lösungsvorschrift, bestehend aus einfachen Operationen und dem Aufruf von Funktionen, ist das Betrachten der gerufenen Funktionen. Weniger wichtig sind einfache Operationen und Zuweisungen. Daher ist für die Laufzeitanalyse die Rekursionstiefe von wichtigster Bedeutung (sieht man z.B. beim Algorithmus von \textsc{Strassen} oder beim Algorithmus \textsc{Merge-Sort}).
	
	Die Rekursionstiefe ist logarithmisch. Kleiner kann die Tiefe für die gesamte Lösung nicht sein, da das Problem mit der 2 im Nenner logarithmisch geteilt wird. Die Funktion, die am stärksten wächst und dabei immer noch eine Umkehrfunktion hat, ist die Exponentialfunktion. Daher ist ihre Umkehrfunktion am schnellsten in der Reduzierung der Rekursionstiefe für die Laufzeitanalyse für Teile-und-Herrsche Algorithmen. Eine Erhöhung des Nenners führt zwangsläufig zu einer Veränderung der anderen beiden Kriterien.
	
	Zu (2), (3): Für die Anzahl der Teilprobleme und für die Größe eines Teilproblems gilt: Angenommen, der Nenner wird vergrößert. Dann wird die ursprüngliche Probleminstanz aber nicht in ihrem leichtesten Punkt geteilt. Denn ein Problem ist nur da am leichtesten zu teilen, wo sein Schwerpunkt liegt. Es gibt für jede Probleminstanz nur einen Schwerpunkt. Wenn der Nenner ganzzahlig verkleinert wird, wird nicht mehr geteilt. Es muss ganzzahlig verkleinert werden, da z.B. ein einzelnes Array-Element nicht teilbar ist. Die Zahl $e$ scheidet damit in der Wahl der Exponentialfunktion aus. Daher muss die Exponentialfunktion 2 als Basis haben.
\end{proof}
Auffällig ist, dass nur das Problem, bei dem zwei Instanzen durch einen Operator verbunden sind, mit dem optimalen Teile-und-Herrsche Prinzip zu lösen ist. Das sind Probleme mit Operatoren wie z.B., $+$, $-$, $<$. Im Prinzip werden diese Probleme an ihrem Schwerpunkt gelöst, wo sie trotz ihrer Schwere am leichtesten zu tragen sind, wie bei einer Wippe mit zwei gleichen Gewichten jeweils am äußersten Ende. Die Wippe ist dann exakt horizontal ausgerichtet im Gleichgewicht.
\begin{lemma}
	Schneller als mit $T(n) = O(n^{2.3219})$ können zwei $n \times n$ Matrizen nicht multipliziert werden.
\end{lemma}
\begin{proof}
	Der Beweis folgt daraus, dass \textsc{Strassen-25}$(A, B)$ zwei $n \times n$ Matrizen nach dem optimalen Teile-und-Herrsche Prinzip multipliziert.
\end{proof}

\begin{satz}
	Wenn es bei einer Lösung erforderlich ist, jedes Element der Eingabe während der Abarbeitung der Lösungsvorschrift einmal zu betrachten, dann gibt es keine bessere Laufzeit als die logarithmische.
\end{satz}
\begin{proof}
	Wenn Laufzeit so aufgefasst wird, dass in jedem Schritt ein Register besucht wird, dann kann die Anzahl der unterschiedlich besuchten Register nicht weniger als logarithmisch sein. Die Sonnenblumenkerne in einer Sonnenblume sind logarithmisch angeordnet und damit optimal untergebracht mit minimal wenig Platz. Sie sind im goldenen Winkel angebracht. Würden sie besser angebracht werden können, dann wäre der goldene Winkel kein goldener Winkel.
\end{proof}

\begin{lemma}
	Das optimale Teile-und-Herrsche Prinzip ist optimal unter allen Lösungsprinzipien, wo es während der Abarbeitung der Lösungsvorschrift erforderlich ist, jedes Element der Eingabe während der Lösung einmal zu betrachten.
\end{lemma}
\begin{proof}
	Der Beweis folgt daraus, dass Algorithmen, die nach dem optimalen Teile-und-Herrsche Prinzip Lösungsvorschriften abarbeiten, logarithmische Laufzeit haben müssen. 
\end{proof}

\subsubsection{Erwartete Ergebnisse}
Bei der Analyse der zu messenden Laufzeit von Algorithmus \textsc{Strassen-25} ist zu erwarten, dass dieser mit nur 5 Matrixmultiplikationen deutlich weniger Rechenzeit benötigt als der Algorithmus von \textsc{Strassen} mit 7 Matrixmultiplikationen.

Die Tabelle \ref{tab:operations-comparison} zeigt die theoretische Anzahl benötigter Operationen für die Matrixmultiplikation in Abhängigkeit von $n$. Die Operationen sind elementare arithmetische Operationen wie Multiplikationen und Additionen/Subtraktionen von Skalaren. Die Anzahl der Operationen in der Tabelle sind asymptotisch und ignorieren konstante Faktoren, um die Skalierungseffekte zu betonen.
\begin{table}[h]
	\centering
	\caption{Vergleich der theoretischen Anzahl an Operationen}
	\label{tab:operations-comparison}
	\renewcommand{\arraystretch}{1.5} % Zeilenhöhe erhöhen
	\begin{tabular}{m{1.5cm}|m{3cm}|m{3cm}|m{3cm}|m{2cm}}
		\hline
		$n$ & Standard & \textsc{Strassen} & \textsc{Strassen-25} & Matrixgröße ($n \times n$) \\
		\hline\hline
		10 & $10^3 = 1.000$ & $10^{2.8074} \approx 642$ & $10^{2.3219} \approx 209$ & $100$ \\
		\hline
		100 & $100^3 = 10^6$ & $100^{2.8074} \approx 6.4 \times 10^5$ & $100^{2.3219} \approx 2.09 \times 10^4$ & $10^4$ \\
		\hline
		1.000 & $1.000^3 = 10^9$ & $1.000^{2.8074} \approx 6.4 \times 10^8$ & $1.000^{2.3219} \approx 2.09 \times 10^7$ & $10^6$ \\
		\hline
		10.000 & $10.000^3 = 10^{12}$ & $10.000^{2.8074} \approx 6.4 \times 10^{11}$ & $10.000^{2.3219} \approx 2.09 \times 10^{9}$ & $10^8$ \\
		\hline
		100.000 & $100.000^3 = 10^{15}$ & $100.000^{2.8074} \approx 6.4 \times 10^{14}$ & $100.000^{2.3219} \approx 2.09 \times 10^{11}$ & $10^{10}$ \\
		\hline
		1.000.000 & $1.000.000^3 = 10^{18}$ & $1.000.000^{2.8074} \approx 6.4 \times 10^{17}$ & $1.000.000^{2.3219} \approx 2.09 \times 10^{13}$ & $10^{12}$ \\
		\hline
		$10^7$ & $10^{21}$ & $6.4 \times 10^{20}$ & $2.09 \times 10^{15}$ & $10^{14}$ \\
		\hline
		$10^8$ & $10^{24}$ & $6.4 \times 10^{23}$ & $2.09 \times 10^{17}$ & $10^{16}$ \\
		\hline
	\end{tabular}
\end{table}
Die Wahl des geeigneten Algorithmus zur Matrixmultiplikation hängt stark von der Größe der Matrizen und dem spezifischen Anwendungsbereich ab. Interessant ist, dass schon für $n = 100$ nur $10^{4}$ Operationen benötigt werden von \textsc{Strassen-25} und dagegen $10^{5}$ Operationen von \textsc{Strassen}. Für $n \geq 10.000$ unterscheiden sich \textsc{Strassen-25} und \textsc{Strassen} um 2 Zehnerpotenzen und für $n \geq 10^8$ unterscheiden sie sich um 6 Zehnerpotenzen. Praktische Anwendungsbereiche für die $n \times n$ Matrixmultiplikation sind die folgenden:

\begin{enumerate}
	\item \textbf{Kleine Matrizen} ($n < 200$): Für kleinere Matrizen, wie sie in der Computergrafik oder bei einfachen linearen Gleichungssystemen vorkommen, ist der Standardalgorithmus mit $O(n^3)$ aufgrund seines geringen zusätzlichen Aufwands für die Rekursion und optimaler Cache-Nutzung die erste Wahl.
	
	\item \textbf{Mittlere bis große Matrizen} ($n \approx 200 - 10.000$): Im wissenschaftlichen Rechnen, bei Optimierungsproblemen oder im maschinellen Lernen dominieren weiterhin optimierte Implementierungen des Standardalgorithmus.
	
	\item \textbf{Sehr große Matrizen} ($n \ge 10.000$): Bei großen Matrizen, wie sie in Big Data Analysen oder beim Training umfangreicher Deep-Learning-Modelle auftreten, wird der Bedarf an Alternativen zu optimierten Standardalgorithmen deutlich. Damit gewinnt der Algorithmus von \textsc{Strassen} an Attraktivität und Relevanz für die Bewältigung rechenintensiver Multiplikationen.
\end{enumerate}

\section{Matrixoperationen}
Eine Überlegung ist die folgende: Halte die Wertebereiche von Elementen einer Matrix in ihrer Größe klein, finde die Menge aller Kerne, mit Hilfe derer Ergebnisse von Matrixoperationen effizienter nachgeschaut werden können. Vergrößere den Wertebereich um ein Delta, das für das effizientere Nachschauen im vergrößerten Wertebereich genutzt wird.

\section{Finden von Kernen}
In der Beweisführung, dass es keinen Teile-und-Herrsche Algorithmus zur Multiplikation von $n \times n$ Matrizen gibt, der weniger als 5 Matrixmultiplikationen benötigt, wurde die Eigenschaft des Projezierens genutzt. Zwei Vektoren $\boldsymbol{u}$ und $\boldsymbol{v}$ projezieren $\boldsymbol{w}$ genau dann, wenn es Konstanten $k_1$ und $k_2$ gibt, so dass $$k_1 \boldsymbol{u} + k_2\boldsymbol{v} = \boldsymbol{w} \neq \boldsymbol{0}.$$ Es ist im Allgemeinen von Interesse, eine minimale Projektion zu finden: Finde eine minimale Projektion, nimm noch eine Konstante $k$ weg, dann ist es die \textit{größte Einheit}, die aus sich heraus, egal wie, nichts mehr projezieren kann außer $\boldsymbol{0}$. Die Idee des Algorithmus von \textsc{Strassen-25} eignet sich zum Finden von größten, nicht projezierbaren Einheiten.

\chapter{Zusammenfassung}
In dieser Arbeit werden die grundlegenden Strukturen der Algebra wie Vektoren und Matrizen definiert. Sie werden als Strukturen verstanden, die durch Verknüpfungen miteinander verbunden sind. Vektoren sind dabei eine besondere Struktur. Sie bilden einen wohl geformten Körper. Es wird gezeigt, dass die charakteristischen Eigenschaften eines wohl geformten Körpers gebildet werden durch: (1) die Verknüpfungen: Addition und Multiplikation und (2) den Wertebereich.

In Kapitel \ref{chap:anwendungen} wird die Matrixmultiplikation und der Algorithmus von \textsc{Strassen} zur effizienten Multiplikation von Matrizen vorgestellt. Der \textsc{Strassen}-Algorithmus multipliziert zwei $n \times n$ Matrizen $A$ und $B$ unter Verwendung von nur 7 Matrixmultiplikationen anstelle der üblichen 8. Die Laufzeit des Algorithmus wird durch die Rekurrenzgleichung $T(n) = 7 \cdot T\left(\frac{n}{2}\right) + c \cdot n^2$ beschrieben. Der Algorithmus von \textsc{Strassen} hat damit eine Laufzeit von $O(n^{2.8074})$, welche analysiert wird. Der Algorithmus wird in einem Pseudocode vorgestellt. Er unterteilt die Matrizen in 4 Teilmatrizen und berechnet 7 Produkte, die dann zur Berechnung der 4 Blöcke der Ergebnismatrix verwendet werden.

Eine Methode zur Reduzierung der Matrixmultiplikationen von 7 auf 5 wird beschrieben, was die Laufzeit auf $O(n^{2.3219})$ senkt. Diese Optimierung nutzt die Diagonale der Matrix $B$ und erfordert vorherige Berechnungen der Produkte. Der Algorithmus \textsc{Strassen-25} setzt diese Methode um. Das \textit{optimale Teile-und-Herrsche Prinzip} wird vorgestellt und gezeigt, dass es die effizienteste Methode ist, um Probleme zu lösen, wenn die Probleminstanz bei jedem rekursiven Aufruf halbiert wird. Es wird gezeigt, dass schneller als mit Laufzeit $O(n^{2.3219})$ zwei $n \times n$ Matrizen \textit{nicht} multipliziert werden können.

\section{Ausblick}

\end{document}